%----------------------------------------------------------------------------------------
%	INTRODUCTION SECTION
%----------------------------------------------------------------------------------------
\section{Introduction}
% Context du stage, que as tu fait, ou, quand, resultats.

This report refers to the work I have completed during my summer internship with the IRIF at  Paris Diderot University - Paris 7 under the supervision of Mihaela Sighireanu. The work performed concern the study  of dynamic memory allocation , programming some allocators and testing their efficiency.\\

Dynamic memory allocation has been a fundamental part of most computer systems since roughly 1960, and memory allocation is widely considered to be either a solved problem or an insoluble one.\\

 In this report I'll describe a variety of Memory allocator design first and then, I will explain how I manage to test my programs.




%----------------------------------------------------------------------------------------
%	Presentation of IRIF SECTION
%----------------------------------------------------------------------------------------

\subsection{Presentation of IRIF}
\textbf{Research Institute on the Foundations of Computer Science }
(IRIF) is a research unit co-founded by the CNRS and the University Paris-Diderot, as UMR 8243. IRIF hosts two INRIA project-teams. IRIF is also member of the Foundation of Mathematical Sciences of Paris (FSMP) and of the Federation of Research in Mathematics of Paris. \\

IRIF results from the merging of the two research units LIAFA and PPS on January 1st, 2016. The scientific objectives of IRIF are at the core of computer science and, in particular, they focus on the conception, analysis, proof, and verification of algorithms, programs, and programming languages. They are built upon fundamental research activities developed at IRIF on combinatorics, graphs, logics, automata, type, semantics, and algebras.\\

At the CNRS, IRIF is primarily affiliated to the Institute of Computer Science (INS2I), and, secondarily, to the Institute of Mathematics (INSMI). IRIF is part of the Department of Computer Science of the University Paris-Diderot, and also hosts several members of the Department of Mathematics. Last, IRIF is affiliated to the doctoral school of Mathematical Sciences of Paris (ED 386).\\ 

IRIF is structured in nine thematic teams, grouped into three research poles: 
\begin{itemize}
\item Pole Algorithms and discrete structures:

\begin{itemize}
\item Algorithms and complexity
\end{itemize}

\begin{itemize}
\item Combinatorics
\end{itemize}

\begin{itemize}
\item Complex Systems, Networks, and Distributed Computing 
\end{itemize}

\begin{itemize}
\item Theory and algorithmics of graphs 
\end{itemize}

\end{itemize}

\begin{itemize}
\item Pole Automata, structures and verification:

\begin{itemize}
\item Automata and applications
\end{itemize}
\begin{itemize}
\item Modeling and verification
\end{itemize}

\end{itemize}

\begin{itemize}
\item Pole Proofs, programs and systems:
\begin{itemize}
\item Algebra and computation
\end{itemize}
\begin{itemize}
\item Analysis and conception of systems 
\end{itemize}
\begin{itemize}
\item Proofs and programs
\end{itemize}

\end{itemize}
IRIF hosts currently a hundred permanent members), including roughly 54 faculty members, 30 CNRS researchers, 6 INRIA researchers and 5 administrative and technical staffs (in January 2018). The total number of IRIF members, including PhD students, post-docs, and long-term visitors amounts for about two hundred. \\

I've done my internship the the ``modeling and verification'' team, which studies and develops methods and tools for the software verification.


%----------------------------------------------------------------------------------------
%	Motivation SECTION
%----------------------------------------------------------------------------------------
\subsection{Motivation}
Allocators are small programs which are difficult to design because they use some primitives of complex programming: arithmetic of pointers, bitwise operations, system calls (for the expansion of process memory).   
Furthermore, data structures are quite complex : heap list (the next field gotten with arithmetic of pointers) , linked and circular lists, binary search trees, hash tables.\\

In the order to implement these programs, formal verification techniques have been developed. Those techniques are based on some logics of programs (logic of Hoare). My work has provided testing programs for these techniques.\\

In my internship I developed some of these allocators using unit testing and the CUNIT library.\\

% les allocateurs sont des programmes petits mais tres difficiles à mettre au point car ils appellent des primitives de programmation complexes: arithmetiques de pointeur, operations bit à bit, appels aux systeme (pour etendre la memoire du processus). De plus, les structures de données sont assez complexes: heap-listes (chnamps next obtenu avec arithmetique de pointeur), listes doublement chainées, circulaires, arbres binaires de rechrche, table de hashage. 

% Pour mettre au point ce programmes, des techniques de verification formelles ont été developpées. Ces techniques sont basées sur des logiques de programmes (logique de Hoare). Mon travail a fournit des programmes de test de ces techniques.


% Dans mon stage j'ai mis au point ces allocateurs en utilisant du test unitaire et la  bibliotheque CUNIT.
%%%%%%%%%%%%%%%%%%%%%%%%%%%%%