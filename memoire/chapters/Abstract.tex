Les allocateurs sont des programmes petits mais très difficiles à mettre au point car ils appellent des primitives de programmation complexes: arithmetique des pointeurs, operations bit à bit, appels au système (pour étendre la memoire du processus). De plus, les structures de données sont assez complexes: heap-listes (champs suivant obtenu avec l'arithmetique des pointeurs), listes doublement chainées, circulaires, arbres binaires de rechrche, tables de hachage. 

Pour mettre au point ces programmes, des techniques de verification formelles ont été developpées. Ces techniques sont basées sur des logiques de programmes (logique de Hoare). Mon travail a fournit des programmes de test de ces techniques.

Dans mon stage j'ai mis au point ces allocateurs en utilisant du test unitaire et la  bibliotheque CUNIT.
%%%%%%%%%%%%%%%%%%%%%%%%%%%%%