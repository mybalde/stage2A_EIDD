\documentclass[11pt]{article}
\usepackage[english]{babel}
\usepackage[utf8x]{inputenc}
\usepackage{amsmath}
\usepackage{graphicx}
\usepackage{float}
\usepackage[colorinlistoftodos]{todonotes}
\usepackage[T1]{fontenc}
\usepackage{listings}
\def\code{\lstinline[basicstyle=\ttfamily]}
%\usepackage[margin=0cm]{geometry}
\usepackage{hyperref}
\usepackage{algorithm}
\usepackage{algorithmic}
\usepackage{xcolor}

\definecolor{mGreen}{rgb}{0,0.6,0}
\definecolor{mGray}{rgb}{0.5,0.5,0.5}
\definecolor{mPurple}{rgb}{0.58,0,0.82}
\definecolor{backgroundColour}{rgb}{0.95,0.95,0.92}

\lstdefinestyle{cstyle}{
    backgroundcolor=\color{backgroundColour},   
    commentstyle=\color{mGreen},
    keywordstyle=\color{magenta},
    numberstyle=\tiny\color{mGray},
    stringstyle=\color{mPurple},
    basicstyle=\footnotesize\ttfamily,
    breakatwhitespace=false,         
    breaklines=true,                 
    captionpos=b,                    
    keepspaces=true,                 
    numbers=left,                    
    numbersep=5pt,                  
    showspaces=false,                
    showstringspaces=false,
    showtabs=false,                  
    tabsize=4,
    language=C
}


\begin{document}

\begin{titlepage}

\newcommand{\HRule}{\rule{\linewidth}{0.5mm}} % Defines a new command for the horizontal lines, change thickness here

\center % Center everything on the page
 
%----------------------------------------------------------------------------------------
%	HEADING SECTIONS
%----------------------------------------------------------------------------------------

\textsc{\LARGE Paris Diderot University}\\[1.5cm]
\textsc{\Large Engineering School Denis Diderot}\\[0.5cm] 

%----------------------------------------------------------------------------------------
%	TITLE SECTION
%----------------------------------------------------------------------------------------


\HRule \\[0.4cm]
{ \huge \bfseries Dynamic Memory Allocation in C}\\[0.4cm] 
\HRule \\[1.5cm]
 
 \textsc{\large second year internship report}\\[1.5cm] 
%----------------------------------------------------------------------------------------
%	AUTHOR SECTION
%----------------------------------------------------------------------------------------

\begin{minipage}{0.4\textwidth}
\begin{flushleft} \large
\emph{Author:}\\
Mamadou Yéro \textsc{Baldé} % Your name
\end{flushleft}
\end{minipage}
~
\begin{minipage}{0.4\textwidth}
\begin{flushright} \large
\emph{Supervisor:} \\
Mihaela \textsc{Sighireanu} % Supervisor's Name
\end{flushright}
\end{minipage}\\[2cm]

%----------------------------------------------------------------------------------------
%	DATE SECTION
%----------------------------------------------------------------------------------------

{\large \today}\\[0.1cm] % Date, change the \today to a set date if you want to be precise

%----------------------------------------------------------------------------------------
%	LOGO SECTION
%----------------------------------------------------------------------------------------

\includegraphics[width=0.5\textwidth]{figures/eidd.jpeg}\\[0cm] % Include a department/university logo - this will require the graphic package
 
%----------------------------------------------------------------------------------------

\vfill % Fill the rest of the page with whitespace

\end{titlepage}

Les allocateurs de memoire dynamique gèrent le tas (\emph{heap}) d'un program.
Ils sont indispensables à tout langage de programmation moderne. En C, l'allocateur de memoire est inclu dans la bibliothèque standard; en Java, il fait partie de l'environnement d'exécution (\emph{runtime}).
Le code de ces allocateurs est assez petit, mais il est notoirement difficile à mettre au point car il doit utiliser des primitives de programmation complexes: arithmetique des pointeurs, operations bit à bit, appels système. De plus, les structures de données que ce code utilise sont assez complexes: liste-tas (champs suivant obtenu avec l'arithmetique des pointeurs), listes doublement chainées, listes circulaires, arbres binaires de recherche, tables de hachage. 

Pour mettre au point ces programmes, des techniques de verification formelles ont été developpées par l'équipe ``Modélisation et Vérification'' de l'IRIF, où mon stage a eu lieu. Ces techniques sont basées sur des logiques de programmes. Mon travail a été de fournir des programmes de test pour ces techniques.
J'ai donc mis au point quatre allocateurs en utilisant des techniques de test unitaire et en s'appuyant sur la  bibliotheque CUNIT.
%%%%%%%%%%%%%%%%%%%%%%%%%%%%%
\clearpage

\tableofcontents

%----------------------------------------------------------------------------------------
%	Report CONTENT - CHAPTERS
%----------------------------------------------------------------------------------------
%----------------------------------------------------------------------------------------
%	INTRODUCTION SECTION
%----------------------------------------------------------------------------------------
\section{Introduction}
% Context du stage, que as tu fait, ou, quand, resultats.
Dynamic memory allocation has been a fundamental part of most computer systems since roughly 1960, and memory allocation is widely considered to be either a solved problem or an insoluble one.\\

This report refers to the work I have completed during my summer internship with the IRIF at  Paris Diderot University - Paris 7 under the supervision of Mihaela Sighireanu. The work performed concern the study  of dynamic memory allocation , programming some allocators and testing their efficiency.\\

 In this report I'll describe a variety of Memory allocator design first and then, I will explain how I manage to test my programs.




%----------------------------------------------------------------------------------------
%	Presentation of IRIF SECTION
%----------------------------------------------------------------------------------------

\subsection{Presentation of IRIF}
\textbf{Research Institute on the Foundations of Computer Science }
(IRIF) is a research unit co-founded by the CNRS and the University Paris-Diderot, as UMR 8243. IRIF hosts two INRIA project-teams. IRIF is also member of the Foundation of Mathematical Sciences of Paris (FSMP) and of the Federation of Research in Mathematics of Paris. \\

IRIF results from the merging of the two research units LIAFA and PPS on January 1st, 2016. The scientific objectives of IRIF are at the core of computer science and, in particular, they focus on the conception, analysis, proof, and verification of algorithms, programs, and programming languages. They are built upon fundamental research activities developed at IRIF on combinatorics, graphs, logics, automata, type, semantics, and algebras.\\

At the CNRS, IRIF is primarily affiliated to the Institute of Computer Science (INS2I), and, secondarily, to the Institute of Mathematics (INSMI). IRIF is part of the Department of Computer Science of the University Paris-Diderot, and also hosts several members of the Department of Mathematics. Last, IRIF is affiliated to the doctoral school of Mathematical Sciences of Paris (ED 386).\\ 

IRIF is structured in nine thematic teams, grouped into three research poles: 
\begin{itemize}
\item Pole Algorithms and discrete structures:

\begin{itemize}
\item Algorithms and complexity
\end{itemize}

\begin{itemize}
\item Combinatorics
\end{itemize}

\begin{itemize}
\item Complex Systems, Networks, and Distributed Computing 
\end{itemize}

\begin{itemize}
\item Theory and algorithmics of graphs 
\end{itemize}

\end{itemize}

\begin{itemize}
\item Pole Automata, structures and verification:

\begin{itemize}
\item Automata and applications
\end{itemize}
\begin{itemize}
\item Modeling and verification
\end{itemize}

\end{itemize}

\begin{itemize}
\item Pole Proofs, programs and systems:
\begin{itemize}
\item Algebra and computation
\end{itemize}
\begin{itemize}
\item Analysis and conception of systems 
\end{itemize}
\begin{itemize}
\item Proofs and programs
\end{itemize}

\end{itemize}
IRIF hosts currently a hundred permanent members), including roughly 54 faculty members, 30 CNRS researchers, 6 INRIA researchers and 5 administrative and technical staffs (in January 2018). The total number of IRIF members, including PhD students, post-docs, and long-term visitors amounts for about two hundred. \\

I've done my internship the the ``modeling and verification'' team, which studies and develops methods and tools for the software verification.


%----------------------------------------------------------------------------------------
%	Motivation SECTION
%----------------------------------------------------------------------------------------
\subsection{Motivation}
Allocators are small programs which are difficult to design because they use some primitives of complex programming: arithmetic of pointers, bitwise operations, system calls (for the expansion of process memory).   
Furthermore, data structures are quite complex : heap list (the next field gotten with arithmetic of pointers) , linked and circular lists, binary search trees, hash tables.\\

In the order to implement these programs, formal verification techniques have been developed. Those techniques are based on some logics of programs (logic of Hoare). My work has provided testing programs for these techniques.\\

In my internship I developed some of these allocators using unit testing and the CUNIT library.\\


\section{Dynamic memory allocation}

Memory allocation is mainly a computer hardware operation which is managed through operating system and software applications. Its process is pretty similar to the physical and virtual memory management. Programs and services are assigned with a specific memory as per their requirements when they are executed. Once the program has finished its operation or is idle, the memory is released and allocated to another program or merged within the primary memory.\\

Memory allocation has two core types:
\begin{itemize}
\item Static Memory Allocation: The program is allocated memory at compile time.
\end{itemize}

\begin{itemize}
\item Dynamic Memory Allocation: The programs are allocated with memory at run time.
\end{itemize}

In this report, we will focus on the dynamic allocation of memory and its operation.

\subsection{User interface}

Dynamic Memory Allocation refers to managing system memory is when an executing program requests that the operating system give it a block of main memory. The program then uses this memory for some purpose. Usually the purpose is to add a node to a data structure. In object oriented languages, dynamic memory allocation is used to get the memory for a new object.\\

The memory comes from above the static part of the data segment. Programs may request memory and may also return previously dynamically allocated memory. Memory may be returned whenever it is no longer needed. Memory can be returned in any order without any relation to the order in which it was allocated. The heap may develop ``holes'' where previously allocated memory has been returned between blocks of memory still in use.\\

A new dynamic request for memory might return a range of addresses out of one of the holes. But it might not use up all the hole, so further dynamic requests might be satisfied out of the original hole.\\

If too many small holes develop, memory is wasted because the total memory used by the holes may be large, but the holes cannot be used to satisfy dynamic requests. This situation is called memory fragmentation \cite{Knuth73a}. Keeping track of allocated and deallocated memory is complicated. A modern operating system does all this.\\

Two functions make it possible to reserve and release dynamically an area of memory: \code{malloc()} for the reservation and the liberation of previously allocated memory via \code{malloc()} is provided by the \code{free()} function.

\subsection{What is malloc}
malloc is a C Standard Library function which allocates memory chunks.
The function signature is:
\begin{lstlisting}
void* malloc(size_t size);
\end{lstlisting}

The only parameter to pass to malloc is the number of bytes to allocate. The returned value is the address of the first byte of the allocated memory area. If the allocation could not be realized (due to lack of free memory), the return value is the NULL constant.

% The two core elements of the malloc algorithm have remained unchanged since the earliest versions:

% \textbf{Boundary Tags}


\subsection{Goal for good allocators}
An allocator must keep tracking parts of the memory which are in use and free.
The main goals of a good allocator are \cite{Lea12}:
\begin{enumerate}
\item Maximizing compatibility\\

\item Maximizing portability

\item Minimizing space

\item Minimizing time
\item Maximizing tunability
\item Maximizing Locality
\item Maximizing error detection
\end{enumerate}

\section{The heap and System calls}
Before writing a first malloc, we need to understand how memory is handled in most operating systems.We will keep an abstract point of view for that part, since many details are system and
hardware dependent.

\subsection{The Process’s Memory}
Each process has its own virtual address space dynamically translated into physical memory  address space by the MMU (and the kernel.) This space is divided in several part, all that we have to know is that we found at least some space for the code, a stack where local and volatile data are stored, some space for constant and global variables and an unorganized space for program’s data called the heap.
The heap is a continuous (in terms of virtual addresses) space of memory with three bounds: a starting point, a maximum limit (managed through \code{sys/resource.h}’s functions \code{getrlimit()} and \code{setrlimit()}) and an end point called the break. The break marks the end of the mapped memory space, that is, the part of the virtual address space that has correspondence into real memory. 
% Figure
% 1
% sketches the memory organisation.

\subsection{Syscall}
In order to code a malloc, we need to know where the heap begin and the break position,
and of course we need to be able to move the break. This the purpose of the two syscalls \code{brk()} and \code{sbrk()}.\\
The functions signature:
\begin{lstlisting}
int brk(const void *addr);
void* sbrk(intptr_t incr);
\end{lstlisting}
\code{brk()} and \code{sbrk()} change the location of the program break, which defines the end of the process's data segment (i.e., the program break is the first location after the end of the uninitialized data segment). Increasing the program break has the effect of allocating memory to the process; decreasing the break deallocates memory.

\code{brk()} sets the end of the data segment to the value specified by addr, when that value is reasonable, the system has enough memory, and the process does not exceed its maximum data size.

\code{sbrk()} increments the program's data space by increment bytes. Calling \code{sbrk()} with an increment of 0 can be used to find the current location of the program break.

On success, \code{brk()} returns zero. On error, -1 is returned, and \code{errno} is set to \code{ENOMEM}.

On success, \code{sbrk()} returns the previous program break. (If the break was increased, then this value is a pointer to the start of the newly allocated memory). On error, \code{(void *) -1} is returned, and \code{errno} is set to \code{ENOMEM}. 

We will use \code{sbrk()} as our main tool to implement malloc. All we have to do is to acquire more space (if needed) to fulfill the query.

\section{The design}
\section{Allocators implementation and tests}
The purpose of unit tests is to verify the expected behavior of a function with respect to a given specification. Here the lack of precise specification forces us to define requirements of these functions, what behaviors are expected or not.

One of the goals of this internship is to conduct a series of tests on the different implementations of the \code{malloc()} and \code{free()} functions. As a result, I implemented 3 allocators in order to perform tests.

After describing the implemented allocator, we will first carry out functional tests and subsequently cover tests (test of instruction and condition, decision or mcdc, this choice will be explained in the following).

The functional tests will consist of a call for each function, and their expected answers. The individual tests will be performed out of any context of use while the test suites will verify a succession of several orders.

\subsection{Leslie Aldrige's allocator}
I began first by trying to understand the allocator written by Leslie Aldrige to get an idea about the implementation of allocators, and I noticed several bugs in his implementation, so I fixed them all, and compiled it to see the results. This implementation is based on the first fit method, using a free list. After that, I implemented three Allocators with Different policies \cite{Aldridge08}.
Here is the revised version of Aldrige's code :\\
Header file
\lstinputlisting[basicstyle=\ttfamily,style=cstyle]{src/lamalloc/leslie.h}
C file
\lstinputlisting[basicstyle=\ttfamily,style=cstyle]{src/lamalloc/leslie.c}
main file
\lstinputlisting[basicstyle=\ttfamily,style=cstyle]{src/lamalloc/user_leslie.c}

\subsection{First version}
This first implementation is based on the first fit method with early coalescing. As soon as a block is released, checks are made to merge the contiguous blocks.

\lstinputlisting[basicstyle=\ttfamily,style=cstyle]{src/mbmalloc/FF_malloc/malloc.h}
\lstinputlisting[basicstyle=\ttfamily,style=cstyle]{src/mbmalloc/FF_malloc/malloc.c}
For the code test, we will first perform tests on the functions in individual ways, out of any context of use and those for each of the functions present.
Then we will perform the tests in a context of use with a succession of function call.
\subsubsection{test of malloc}
In this test, we try to do some kind of dynamic allocation to make sure that it works properly and then we check that the heap start and the break are not \code{NULL}.

\subsubsection{test of free}
In this test, we try to do some dynamic allocation and we release the memory and then we try to see the state of the block to ensure the proper functioning of the function.

\subsubsection{Statement Coverage}
In this test, we try to check all the main lines of the code. We first check that heap start and heap end are \code{NULL} before making an allocation then we check that heap start and heap end are not \code{NULL} and we release the allocated memory.
\subsubsection{MC/DC cover}
In software testing, the modified condition/decision coverage (MC/DC) is a code coverage criterion that requires all of the below during testing:

 \begin{enumerate}
\item Each entry and exit point is invoked
\item Each decision takes every possible outcome
\item Each condition in a decision takes every possible outcome
\item Each condition in a decision is shown to independently affect the outcome of the decision.
\end{enumerate}

In this test, I made several malloc and free to make sure that the merge blocks, the algorithm used and all the conditions were respected.

\lstinputlisting[basicstyle=\ttfamily,style=cstyle]{src/mbmalloc/FF_malloc/main.c}

\subsection{Second version}
This second implementation is based on the best fit method with early coalescing. As soon as a block is released, checks are made to merge the contiguous blocks.
\lstinputlisting[basicstyle=\ttfamily,style=cstyle]{src/mbmalloc/BF_malloc/malloc.h}
\lstinputlisting[basicstyle=\ttfamily,style=cstyle]{src/mbmalloc/BF_malloc/malloc.c}
The tests remain the same as for the first implementation.
\lstinputlisting[basicstyle=\ttfamily,style=cstyle]{src/mbmalloc/BF_malloc/main.c}

\subsection{Third version}
This third implementation is based on the first fit method with a free list. 
\lstinputlisting[basicstyle=\ttfamily,style=cstyle]{src/mbmalloc/freelist_malloc/malloc.h}
\lstinputlisting[basicstyle=\ttfamily,style=cstyle]{src/mbmalloc/freelist_malloc/malloc.c}
The tests remain the same as for the first implementation.


\lstinputlisting[basicstyle=\ttfamily,style=cstyle]{src/mbmalloc/freelist_malloc/main.c}


\section{Conclusion}

This internship was an opportunity to revise C programming and the notions learnt during the ``Operating Systems'' course.
In addition, I've learnt how to write unit tests using CUNIT library.
I also learnt how to write documents using \LaTeX.


\bibliographystyle{plain}
\bibliography{bibli}

\newpage
\appendix
\section{Code for Leslie Aldrige's Allocator}
\label{app:leslie}

Here is the revised version of Aldrige's code:

\paragraph{File \code{leslie.h}}
\lstinputlisting[basicstyle=\ttfamily,style=cstyle]{src/lamalloc/leslie.h}

\paragraph{File \code{leslie.c}}
\lstinputlisting[basicstyle=\ttfamily,style=cstyle]{src/lamalloc/leslie.c}

\paragraph{File \code{user_leslie.c}}
\lstinputlisting[basicstyle=\ttfamily,style=cstyle]{src/lamalloc/user_leslie.c}



\section{Code for the first version}
\label{app:first}

\paragraph{File \code{malloc.h}}
\lstinputlisting[basicstyle=\ttfamily,style=cstyle]{src/mbmalloc/FF_malloc/malloc.h}

\paragraph{File \code{malloc.c}}
\lstinputlisting[basicstyle=\ttfamily,style=cstyle]{src/mbmalloc/FF_malloc/malloc.c}

\paragraph{Unit tests}
\lstinputlisting[basicstyle=\ttfamily,style=cstyle]{src/mbmalloc/FF_malloc/main.c}




\section{Code for the second version}
\label{app:second}

\paragraph{File \code{malloc.h}}
\lstinputlisting[basicstyle=\ttfamily,style=cstyle]{src/mbmalloc/BF_malloc/malloc.h}

\paragraph{File \code{malloc.c}}
\lstinputlisting[basicstyle=\ttfamily,style=cstyle]{src/mbmalloc/BF_malloc/malloc.c}

\paragraph{Unit tests}
\lstinputlisting[basicstyle=\ttfamily,style=cstyle]{src/mbmalloc/BF_malloc/main.c}




\section{Code for the third version}
\label{app:third}

\paragraph{File \code{malloc.h}}
\lstinputlisting[basicstyle=\ttfamily,style=cstyle]{src/mbmalloc/freelist_malloc/malloc.h}

\paragraph{File \code{malloc.c}}
\lstinputlisting[basicstyle=\ttfamily,style=cstyle]{src/mbmalloc/freelist_malloc/malloc.c}



\end{document}